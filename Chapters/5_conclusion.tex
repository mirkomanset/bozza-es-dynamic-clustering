\section{Conclusions}\label{sec:conclusions}

This thesis explored dynamic clustering in non-stationary data streams, where
traditional static methods cannot adapt to evolving distributions and recurring
structures. A modular and adaptive framework was developed to maintain
temporally coherent and interpretable clusters, supported by a set of
transitions, \emph{appearance}, \emph{disappearance}, \emph{survival},
\emph{merge}, \emph{split}, and long-term patterns such as
\emph{reappearance}, \emph{remerge}, and \emph{resplit}. Novel overlap metrics,
including the \textsc{Spherical Overlapping Score (SOS)} and the \textsc{Effective
    Overlapping Score (EffectiveOS)}, provide geometry-aware criteria to robustly
detect cluster transitions, enhancing interpretability and robustness under
gradual drift or recurring patterns.

The framework supports incremental updates, memory-efficient summarization,
visualization, and cluster-level reporting. Experiments demonstrated its
effectiveness in capturing evolving cluster structures over time and providing
human-interpretable insights into data dynamics.

Future developments include extending the framework to density-based clustering
algorithms (e.g., \textsc{DBSCAN}, \textsc{HDBSCAN}) by designing new overlap metrics suited to
non-centroid clusters. The modular architecture allows integration of
alternative clustering methods and distance functions, as well as advanced
triggering strategies informed by drift detection. Another promising direction
is forecasting cluster transitions, leveraging the temporal metadata produced
by the framework to anticipate structural changes and enable timely adaptation
in dynamic data streams.
