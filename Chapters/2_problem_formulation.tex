\section{Problem Formulation}\label{sec:problem_formulation}

This thesis addresses the problem of clustering in evolving data streams, where
the underlying distributions are non-stationary and subject to drift over time.
Traditional clustering methods assume access to fixed datasets and static
distributions, making them unsuitable for environments where data arrives
continuously and patterns evolve. The aim is to establish a framework that can
process data incrementally while also tracking and interpreting how clusters
change across time. In this context, dynamic clustering also serves as a tool
for \emph{drift explainability}, as shifts in cluster structure provide
interpretable evidence of when and how the underlying distribution changes.

\subsection{Streaming Clustering}\label{subsec:prob_streaming_clustering}

A data stream can be described as a potentially infinite sequence of
$d$-dimensional observations $\mathcal{X}=\{\mathbf{x}^t\}_{t=1}^{\infty}$. The
task of streaming clustering is to incrementally partition these observations
into coherent groups while meeting constraints on memory and computation. At
each time $t$, the algorithm produces a clustering
$C^t=\{C_1^t,\dots,C_{k_t}^t\}$, where the number of clusters $k_t$ and their
composition may vary as new data arrives.

Unlike static clustering, streaming clustering must adapt to non-stationary
conditions. Distributions often evolve gradually rather than abruptly, and
patterns may recur over time, reflecting realistic phenomena such as seasonal
effects or user behavior. A useful abstraction is that of \emph{piecewise
    regular drift}, where the stream consists of consecutive phases within which
distributional change is smooth, while transitions between phases are bounded.
This perspective avoids both unrealistic assumptions of perfect stability and
the intractability of arbitrary change. Streaming clustering algorithms must
therefore balance two goals: producing stable partitions within phases while
remaining responsive to evolving structure.

It is important to note that streaming clustering itself only produces
clusterings at successive time steps. It does not, by default, establish
correspondences between clusters across time. The separate task of
\emph{tracking clusters} is required to map clusters between time points and to
interpret how they evolve.

\subsection{Tracking Clusters}\label{subsec:prob_tracking_clusters}

As data distributions shift, the structure of clusters also changes. The
clustering $C^t$ at time $t$ may differ substantially from $C^\tau$ at a later
time $\tau$, requiring mechanisms to model and track transitions. Typical
events include the appearance and disappearance of clusters, as well as their
splitting, merging, or persistence over time. These transitions provide a
structural description of how patterns in the data evolve.

To formalize transitions, clusters at different times are compared by measuring
their degree of \emph{overlap}. We denote by $C_i^t \circledast C_j^\tau$ that
cluster $C_i^t$ at time $t$ and cluster $C_j^\tau$ at time $\tau$ are related,
i.e., their overlap exceeds a given threshold. Depending on the representation,
overlap can be defined via shared elements, similarity of centroids, or
proximity of cluster boundaries. This principle allows consistent mapping of
clusters across time and supports the identification of cluster transitions~\cite{mec}.

Formally, the main types of transitions are defined as follows:

\begin{itemize}
    \item \textbf{Appearance:} $\emptyset^t \rightarrow C_j^\tau$ — A new cluster $C_j^\tau$ appears if
              {\small
                  \begin{equation*}
                      \nexists\, C_i^t \in C^t : C_i^t \circledast C_j^\tau
                  \end{equation*}
              }

    \item \textbf{Disappearance:} $C_i^t \rightarrow \emptyset^\tau$ — A cluster $C_i^t$ disappears if
              {\small
                  \begin{equation*}
                      \nexists\, C_j^\tau \in C^\tau : C_i^t \circledast C_j^\tau
                  \end{equation*}
              }

    \item \textbf{Split:} $C_i^t \rightarrow \{C_1^\tau, \dots, C_v^\tau\}$ — A cluster $C_i^t$ splits into $\{C_1^\tau, \dots, C_v^\tau\}$ if
              {\small
                  \begin{equation*}
                      \forall j \in \{1,\dots,v\}: C_i^t \circledast C_j^\tau
                  \end{equation*}
              }

    \item \textbf{Merge:} $\{C_1^t, \dots, C_u^t\} \rightarrow C_j^\tau$ — Clusters $\{C_1^t, \dots, C_u^t\}$ merge into $C_j^\tau$ if
              {\small
                  \begin{equation*}
                      \forall i \in \{1,\dots,u\}: C_i^t \circledast C_j^\tau
                  \end{equation*}
              }

    \item \textbf{Survival:} $C_i^t \rightarrow C_j^\tau$ — A cluster $C_i^t$ survives as $C_j^\tau$ if
              {\footnotesize
                  \begin{equation*}
                      \begin{aligned}
                          C_i^t \circledast C_j^\tau \;\land\;
                           & \nexists\, C_u^t \in C^t \setminus \{C_i^t\}: C_u^t \circledast C_j^\tau \;\land  \\
                           & \nexists\, C_v^\tau \in C^\tau \setminus \{C_j^\tau\}: C_i^t \circledast C_v^\tau
                      \end{aligned}
                  \end{equation*}
              }
\end{itemize}

Tracking clusters in dynamic streams presents several challenges.
Distributional drift may be gradual or abrupt, making it difficult to decide
when two clusters should be linked. Noise and outliers may obscure genuine
transitions, while overlapping clusters can create ambiguity. Furthermore, the
absence of ground truth in streaming settings complicates evaluation, requiring
surrogate measures or application-driven validation.

\subsection{Dynamic Clustering}\label{subsec:dynamic_clustering}

Dynamic clustering integrates streaming clustering with temporal cluster
tracking, offering a unified framework for understanding the evolution of data
distributions. The objective is not only to efficiently cluster data as it
arrives, but also to interpret how clusters evolve over time. This dual
perspective provides richer insights than static methods, especially in domains
where understanding dynamics is as valuable as supporting informed decisions.

Several analytical goals motivate this framework. First, by observing the
evolution of clusters, it becomes possible to explain when and how drift occurs
and to link these changes to external factors. Second, monitoring the lifecycle
of clusters supports concept evolution analysis, where patterns emerge,
stabilize, and eventually disappear. Third, sudden transitions can serve as
indicators of anomalies or significant events in the underlying data
distributions. Finally, dynamic clustering enhances interpretability by
enabling visualization of cluster trajectories or timelines, helping experts
understand and interpret complex temporal patterns.

In summary, dynamic clustering treats clustering as an evolving process rather
than a one-off computation. It unites the efficiency of online algorithms with
the interpretability required to analyze changes in non-stationary
environments. This formulation provides the foundation for adaptive and
trustworthy clustering systems that can operate effectively under continuous
change.

