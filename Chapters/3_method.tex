\section{Method}\label{method}

This section presents the proposed framework for dynamic clustering in evolving
data streams. The method is designed to capture both short-term structural
changes and long-term recurrence patterns by extending cluster transition
modeling, introducing new similarity measures, and integrating them into a
modular clustering architecture.

\subsection{New Transition Types}
This work extends traditional cluster evolution events by introducing three new
recurrence-aware transitions: \textbf{reappearance} (a previously disappeared
cluster re-emerges), \textbf{remerge} (previously split clusters recombine),
and \textbf{resplit} (previously merged clusters split again).

Cluster evolution is modeled using a graph-based representation, where nodes
correspond to clusters at different timestamps and edges indicate overlap
relationships. This structure enables interpretable identification of both
classical transitions (appearance, disappearance, split, merge) and the newly
introduced recurrence transitions, allowing a more realistic modeling of
evolving data streams over extended periods.

\subsection{Overlapping Scores}
To quantify cluster similarity and guide transition detection, two overlapping
scores are introduced.

\paragraph{Spherical Overlapping Score (SOS).}
For approximately spherical clusters $C_i$ and $C_j$, such as those from
K-means, with centers $\boldsymbol{\mu}_i, \boldsymbol{\mu}_j$ and radii $r_i,
    r_j$, the SOS is defined as:
\[
    \text{SOS}(C_i, C_j) = 2^{- \frac{||\boldsymbol{\mu}_i - \boldsymbol{\mu}_j||_2}{r_i + r_j}}.
\]
This value ranges between 0 and 1, decays smoothly with distance, and naturally
defines a threshold at 0.5 to indicate potential evolutionary continuity.

\paragraph{Effective Overlapping Score (EffectiveOS).}
For clusters modeled as Gaussian distributions with means $\boldsymbol{\mu}_i,
    \boldsymbol{\mu}_j$ and covariances $\Sigma_i, \Sigma_j$, EffectiveOS
generalizes SOS by employing the Mahalanobis distance:

{\scriptsize
\begin{align}
    \text{EffectiveOS}(C_i, C_j)                & = 2^{- \frac{d_M(\boldsymbol{\mu}_i, \boldsymbol{\mu}_j)}{\text{EffSize}_i + \text{EffSize}_j}},                    \\
    d_M(\boldsymbol{\mu}_i, \boldsymbol{\mu}_j) & = \sqrt{(\boldsymbol{\mu}_i - \boldsymbol{\mu}_j)^\top \Sigma_{ij}^{-1} (\boldsymbol{\mu}_i - \boldsymbol{\mu}_j)}, \\
    \Sigma_{ij}                                 & = \frac{\Sigma_i + \Sigma_j}{2}.
\end{align}}

Here, $\text{EffSize}_i$ is a confidence-aware effective size based on
principal variances:
\[
    \text{EffSize}_i = \sqrt{\frac{1}{d} \sum_{\ell=1}^{d} (\chi^2_\alpha \sqrt{\lambda_{\ell,i}})^2},
\]
where $\lambda_{\ell,i}$ are the eigenvalues of $\Sigma_i$ and $\chi^2_\alpha$
is the chi-squared critical value at confidence $\alpha$. EffectiveOS remains
bounded in $(0,1]$ while capturing directional variability and correlation in
ellipsoidal clusters.

\subsection{Dynamic Clustering Framework}
The proposed framework is modular, consisting of five integrated components.

\paragraph{Online phase (micro-clustering).}
Streaming data is summarized into compact statistical representations in real
time. In this work, \textit{CluStream}~\cite{clustream} is employed to ensure
efficiency under memory and latency constraints while retaining essential
structure.

\paragraph{Triggering strategy.}  
A strategy determines when to perform macro-clustering, based either on fixed
time intervals or on data-driven signals that indicate significant
distributional change. For simplicity, and to remain coherent with the original
\textit{CluStream} design, we adopt fixed intervals as the triggering mechanism
in this work.

\paragraph{Offline phase (macro-clustering).}
Micro-clusters are aggregated into higher-level groups. Gaussian Mixture Models
provide flexible representations in which the number of clusters can adapt
dynamically to capture complex evolving structures.

\paragraph{History module.}
Minimal descriptors of past clusters are maintained to track long-term
dynamics. This enables the detection of recurrence patterns such as
\emph{reappearance}, \emph{remerge}, and \emph{resplit}.

\paragraph{Tracking module.}
Macro-clusters across different time steps are compared using similarity
measures such as SOS or EffectiveOS. This facilitates the identification of
structural changes and the reconstruction of historical trajectories.

The framework produces a comprehensive report, including graph-based
visualizations of cluster transitions, spatial trajectories of macro-cluster
evolution, and representative data samples.

\begin{figure}[H]
    \centering
    \includegraphics[width=1\linewidth]{Images/my_images/architecture.png}
    \caption{Proposed architecture for the dynamic clustering framework.}
\end{figure}
