\section{Introduction}\label{ch:introduction}

Machine Learning (ML) is increasingly applied in environments where data arrive
continuously in a streaming manner. Such streaming data often exhibit evolving
statistical properties, a phenomenon known as \emph{data drift}, which poses
challenges for traditional static analysis methods.

Dynamic clustering addresses this issue by incrementally grouping incoming data
and tracking how clusters evolve over time. Unlike conventional approaches, it
focuses on explaining the structural transformations induced by drift rather
than merely detecting it. This provides a human-interpretable view of pattern
evolution, enabling analysts to understand the formation, splitting, merging,
and recurrence of clusters.

By continuously analyzing temporal dynamics, dynamic clustering turns raw,
non-stationary data into actionable insights. It supports informed
decision-making in real-time applications by revealing how and why clusters
change, offering a transparent understanding of evolving data streams.
