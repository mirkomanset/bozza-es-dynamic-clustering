\section{Experiments}\label{sec:experiments}

This section summarizes the evaluation of the proposed dynamic clustering
algorithm on two datasets: a typographical drift dataset derived from the 20
Newsgroups corpus and a real image dataset of fruits. These experiments
highlight the algorithm's ability to adapt to evolving data distributions for
both text and image data.

\subsection{Typographical Drift Dataset}\label{subsec:typographical_drift}

To assess robustness under linguistic noise, a subset of the 20 Newsgroups
dataset was used, consisting of four categories: \emph{rec.autos},
\emph{comp.graphics}, \emph{sci.med}, and \emph{rec.sport.baseball}. Initially,
clean documents were sampled to form a reference distribution. Gradually,
typographical noise was introduced through character-level modifications such
as insertions, deletions, replacements, and swaps, simulating progressive
degradation of text quality.

Each document was embedded into a 384-dimensional semantic vector using the
\texttt{all-MiniLM-L6-v2} transformer model, and reduced to 32 dimensions using
UMAP. The algorithm was configured with $q=200$ maximum number of
micro-clusters, a streaming window $\Delta=100$, a macro-clustering interval
$I=100$, an overlapping threshold $\varepsilon=0.5$ and a confidence level
$\alpha=0.9$.

\input{figures/text_typos_rsamples.tex}

\begin{figure}[H]
    \centering
    \includegraphics[width=1\linewidth]{my_images/experiment_results/text_typos/graph.png}
    \caption{Graphical evolution of Typographical Drift clusters over time.}
\end{figure}

Over the course of the experiment, the algorithm tracked the evolution of
clusters in response to increasing degradation. Initially, the drift manifested as
subtle splits from existing clusters, with merging and re-splitting observed as
the text gradually degraded. When the text became heavily corrupted, a fully
distinct cluster emerged, reflecting marked shifts in the data
distribution. These results demonstrate the algorithm's sensitivity to both
gradual and substantial changes, capturing the temporal dynamics in textual
data.

% \begin{figure}[H]
%     \centering
%     \includegraphics[width=1\linewidth]{my_images/experiment_results/text_typos/selected_spatial.png}
%     \caption{Spatial evolution of Typographical Drift clusters showing temporal progression.}
% \end{figure}

\subsection{Fruits Dataset}\label{subsec:fruits}

The Fruits dataset consists of 2,420 RGB images of tangerines, resized to $100
    \times 100$ pixels. Background segmentation was applied using GrabCut, and
high-dimensional embeddings were extracted using a pretrained ResNet-50
network. The embeddings were reduced to 64 dimensions with UMAP.
The first 200 images served as reference data, and the remaining 2,220 images
were progressively ingested. Hyperparameters were set to $q=200$, $\Delta=50$,
$I=50$, $\varepsilon=0.5$ and $\alpha=0.9$.

\begin{figure}[H]
    \centering
    \includegraphics[width=1\linewidth]{my_images/experiment_results/fruits/graph.png}
    \caption{Graphical evolution of Fruit Images clusters over time.}
\end{figure}

\begin{figure}[H]
    \centering
    \includegraphics[width=1\linewidth]{my_images/experiment_results/fruits/selected_rsamples.png}
    \caption{Representative Samples of clusters of Fruit Images before and after the split.}
\end{figure}

\begin{figure}[H]
    \centering
    \includegraphics[width=1\linewidth]{my_images/experiment_results/fruits/selected_spatial.png}
    \caption{Spatial evolution of clusters of Fruit Images before and after the split.}
\end{figure}

The algorithm successfully captured the emergence of new tangerine types and
gradual visual changes over time. Clusters reflected to distinct
characteristics such as color (green, yellow, orange) and shape (round or
oval). While minor splits were occasionally missed due to sensitivity
thresholds, major transitions, including the appearance of orange tangerines,
were clearly detected.
