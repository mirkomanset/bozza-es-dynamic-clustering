\section{Experiments}\label{sec:experiments}

This section summarizes the evaluation of the proposed dynamic clustering
algorithm on two datasets: a typographical drift dataset derived from the 20
Newsgroups corpus and a real image dataset of fruits. These experiments highlight the
algorithm's ability to adapt to evolving data distributions for both text and
image data.

\subsection{Typographical Drift Dataset}\label{subsec:typographical_drift}

To assess robustness under linguistic noise, a subset of the 20
Newsgroups~\cite{20newsgroups} dataset was used, consisting of four categories:
\emph{rec.autos}, \emph{comp.graphics}, \emph{sci.med}, and
\emph{rec.sport.baseball}. Initially, clean documents were sampled to form a
reference distribution. Gradually, typographical noise was introduced through
character-level modifications such as insertions, deletions, replacements, and
swaps, simulating progressive degradation of text quality.

Each document was embedded into a 384-dimensional semantic vector using the
\texttt{all-MiniLM-L6-v2} transformer model~\cite{sentence-transformers}, and
reduced to 32 dimensions using UMAP~\cite{umap}. The algorithm was configured
with $q=200$ micro-clusters, a streaming window $\Delta=100$, a
macro-clustering interval $I=100$, an overlapping threshold $\varepsilon=0.5$
and a confidence level $\alpha=0.9$.

\begin{table}[ht]
    \centering
    \scriptsize % small font
    \renewcommand{\arraystretch}{0.9} % reduces row height
    \setlength{\tabcolsep}{4pt} % reduces space between columns
    \begin{tabularx}{\linewidth}{c|l|X}
        \hline
        \textbf{Cluster} & \textbf{Interpretation} & \textbf{Text Sample} \\
        \hline
        0 & sci.med & \emph{Ihf antne has any informatiin on thi deficiency would veyr ggreatly apprecaite a response hereq or preferably by Emaul. All I know at tihs opin is a defjciencg can caue smyoglovin to be releasfd,a dn in times of stresz ad hgh ambivent temperatursy could caussez grenal failure. x} \\
        \hline
        1 & rec.sport.baseball & \emph{Bobby Bonilla supposedly use the word 'faggot' when he got mad at that author in the clubhouse. Should he be banned from baseball for a year like Schott?} \\
        \hline
        3 & comp.graphics & \emph{Ihf antne has a ljst if cimpanies oing dagta isualizatoing (software or ahdwzre)I woild liek toheapzrf rom tyem. Thank.s - - rs --} \\
        \hline
        2 & rec.autos & \emph{Cup holders (driving is an importantant enough undertaking) Cellular phones and mobile fax machines (see above) Vanity mirrors on the driver's side. Ashtrays (smokers seem to think it's just fine to use the road) Fake convertible roofs and vinyl roofs. Any gold trim.} \\
        \hline
        6 & Typos & \emph{Thhe qystion is nkt wgethfe your adio wigll e stolen. Tghe question isw en your ryaeio will bestolen.} \\
        \hline
    \end{tabularx}
    \caption{Final representative samples of text typo clusters.}
\end{table}


\begin{figure}[H]
    \centering
    \includegraphics[width=1\linewidth]{my_images/experiment_results/text_typos/graph.png}
    \caption{Graphical evolution of Typographical Drift clusters over time.}
\end{figure}

Over the course of the experiment, the algorithm tracked the evolution of
clusters in response to increasing noise. Initially, the drift manifested as
subtle splits from existing clusters, with merging and re-splitting observed as
the text gradually degraded. When the text became heavily corrupted, fully
distinct clusters emerged, reflecting pronounced shifts in the data
distribution. These results demonstrate the algorithm's sensitivity to both
gradual and substantial changes, capturing complex temporal dynamics in textual
data.

% \begin{figure}[H]
%     \centering
%     \includegraphics[width=1\linewidth]{my_images/experiment_results/text_typos/selected_spatial.png}
%     \caption{Spatial evolution of Typographical Drift clusters showing temporal progression.}
% \end{figure}

\subsection{Fruits Dataset}\label{subsec:fruits}

The Fruits dataset consists of 2,420 RGB images of tangerines, resized to $100
    \times 100$ pixels. Background segmentation was applied using
GrabCut~\cite{grabcut}, and high-dimensional embeddings were extracted using a
pretrained ResNet-50~\cite{resnet} network. The embeddings were reduced to 64
dimensions with UMAP~\cite{umap}. The first 200 images served as reference
data, and the remaining 2,220 images were progressively ingested.
Hyperparameters were set to $q=200$, $\Delta=50$, $I=50$, $\varepsilon=0.5$ and
$\alpha=0.9$.

\begin{figure}[H]
    \centering
    \includegraphics[width=1\linewidth]{my_images/experiment_results/fruits/selected_rsamples.png}
    \caption{Representative Samples of clusters of Fruit Images.}
\end{figure}

\begin{figure}[H]
    \centering
    \includegraphics[width=1\linewidth]{my_images/experiment_results/fruits/selected_spatial.png}
    \caption{Spatial evolution of clusters of Fruit Images showing temporal progression.}
\end{figure}

\begin{figure}[H]
    \centering
    \includegraphics[width=1\linewidth]{my_images/experiment_results/fruits/graph.png}
    \caption{Graphical evolution of Fruit Images clusters over time.}
\end{figure}

The algorithm successfully captured the emergence of new tangerine types and
gradual visual changes over time. Clusters corresponded to distinct
characteristics such as color (green, yellow, orange) and shape (round or
oval). While minor splits were occasionally missed due to sensitivity
thresholds, major transitions, including the appearance of orange tangerines,
were clearly detected.
